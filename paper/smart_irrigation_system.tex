\documentclass[conference]{IEEEtran}
\IEEEoverridecommandlockouts
% The preceding line is only needed to identify funding in the first footnote. If that is unneeded, please comment it out.
\usepackage{cite}
\usepackage{amsmath,amssymb,amsfonts}
\usepackage{algorithmic}
\usepackage{textcomp}
\usepackage{xcolor}
\usepackage{colortbl}
\usepackage[rightcaption]{sidecap}
\usepackage{wrapfig}
\usepackage{adjustbox}
\usepackage{hyperref}
\def\UrlBreaks{\do\/\do-}
\usepackage{graphicx} %package to manage images
\graphicspath{ {./images/} }


\def\BibTeX{{\rm B\kern-.05em{\sc i\kern-.025em b}\kern-.08em
    T\kern-.1667em\lower.7ex\hbox{E}\kern-.125emX}}
\begin{document}

\title{Sistema de Rega Inteligente}

\author{\IEEEauthorblockN{1\textsuperscript{st} Tomás Marcos}
\IEEEauthorblockA{\textit{Faculdade de Ciências Exatas e da Engenharia} \\
\textit{Universidade da Madeira}\\
Funchal, Portugal \\
2037017@student.uma.pt}
\and
\IEEEauthorblockN{2\textsuperscript{nd} Nelson Vieira}
\IEEEauthorblockA{\textit{Faculdade de Ciências Exatas e da Engenharia} \\
\textit{Universidade da Madeira}\\
Funchal, Portugal \\
2080511@student.uma.pt}
}

\maketitle

\begin{abstract}
A água é um recurso precioso, considerado um dos bens essenciais para a vida. 
No entanto, cada vez mais, ouve-se que é um recurso escasso e que rapidamente 
está a se esgotar. A água é utilizada para muitas atividades, sejam elas industriais, 
comerciais ou de lazer. Existem muitas iniciativas que pretendem reduzir o 
consumo e o desperdício de água. Pretendemos explorar um sistema de rega 
inteligente que utilize sensores de forma a reduzir a quantidade de água que 
é utilizada. \\
\end{abstract}

\begin{IEEEkeywords}
IoT, Computação ubíqua, Rega inteligente, Análise literária.
\end{IEEEkeywords}

\section{Introdução}
\IEEEPARstart{A} sustentabilidade global não será alcançada sem garantir a 
disponibilidade de água preciosa para todos os consumidores. Apesar de ser um 
dos principais objetivos da agenda da UN2030 \cite{un2015agenda} para o desenvolvimento 
global sustentável, a atual escassez de água está a crescer rapidamente e 
afetando um número crescente de consumidores de água residencial, comercial, 
industrial e agrícola em todo o mundo \cite{mishra2021water}. Espera-se que a 
procura global da água suba 55\%, enquanto atualmente, cerca de 25\% das grandes 
cidades estão a passar por alguns níveis de stress hídrico \cite{josefine2021differentiated}.

As mudanças climáticas, secas graves, crescimento populacional, aumento da 
procura e má administração durante as últimas décadas enfatizaram ainda mais 
os recursos escassos da água doce em todo o mundo e resultaram numa grave 
escassez de água para cerca de 4 bilhões de pessoas, pelo menos um mês 
anualmente \cite{jafari2018assessing} \cite{unicef2019progress} \cite{orimoloye2021spatial}. \cite{salehi2022global}

Um dos setores de atividade humana que tem maior consumo dos recursos hídricos 
é a agricultura, "aproximadamente 100 vezes mais do que o uso pessoal é consumida 
pela alimentação e agricultura e quase 70\% das águas fluviais e subterrâneas 
são utilizadas na irrigação". \cite{nawandar2019iot} Apenas 17\% das terras 
agrícolas, de todo o mundo são irrigadas. Apesar de por todo o mundo, se verificar 
um aumento de terrenos irrigados, a área irrigada per capita tem estado a diminuir 
desde 1990 devido ao rápido crescimento global. \cite{pimentelwater}

Existem, atualmente, várias tecnologias de irrigação, sendo as mais comuns 
a irrigação por inundação e irrigação por aspersão. Outros métodos mais 
focados, como a irrigação gota-a-gota têm maior eficiência hídrica. Esta 
técnica usa uma quantidade inferior de água, entre 30\% a 50\%, quando 
comparada com uma técnica de irrigação superficial. \cite{pimentelwater}

Várias iniciativas foram tomadas para ajudar a minimizar o desperdício 
de água neste setor, mas, no entanto, não aparentam ter muito sucesso, 
ou não são apelativas, devido aos elevados custos associados. 
Os sensores comerciais para sistemas destinados à agricultura e à sua 
irrigação são muito caros, tornando impossível aos pequenos agricultores 
a implementação deste tipo de sistema nas suas explorações. No entanto, 
os fabricantes oferecem actualmente sensores de baixo custo que podem 
ser ligados a nós para implementar sistemas de baixo custo para a gestão da 
irrigação e monitorização agrícola. Além disso, devido ao interesse em 
sensores de baixo custo para monitorizar a agricultura e a água, 
novos sensores de baixo custo estão a ser propostos em vários estudos. \cite{garcia2020iot}

Por estes motivos é importante gerir o consumo de água no nosso dia a dia, 
portanto o que propomos é um sistema de rega inteligente que faz a 
medição da humidade do solo e rega as plantas apenas durante o tempo 
necessário poupando o gasto desnecessário da água de rega.

\section{Trabalhos Relacionados}
Segundo um estudo realizado por García et al, existem 178 artigos 
relacionados com  "IoT irrigation, IoT irrigation system, and smart 
irrigation" \cite{garcia2020iot}, escritos em Inglês, no período de entre os 
anos de 2014 e 2019, inclusive, dos quais 106 artigos estão relacionados com a 
utilização de sensores para monitorizar o estado do solo. Destes 106 artigos 
estudados, todos os artigos abordam a humidade do solo, 9 discutem a temperatura 
do solo, 4 exploram o ph do solo e 3 mencionam os nutrientes presentes no solo.

Dos artigos que mencionam o tipo de sensor utilizado, o sensor mais popular é o 
e YL69 (SparkFun Electronics, Niwot, CO, USA). Este sensor tem um baixo custo e 
foi criado para operar especificamente com o Arduino. \cite{garcia2020iot}

No estudo realizado por Abbas et al. \cite{abbas2014smart}, os autores propõem 
um sistema de irrigação inteligente no qual utilizaram uma rede de sensores sem 
fios para detetar a humidade no solo. Um dos focos do estudo proposto foi a medição 
do tempo de resposta e a capacidade do sistema identificar a capacidade de 
retenção de água do tipo de solo no qual os sensores estavam localizados. 
No entanto, os autores referem que seria necessário ter em consideração outros 
aspetos, tais como a estação do ano, pois no Verão, é necessário regar as plantas 
com maior freqência.

\subsection{Caso Relacionado}

Muitos sistemas de rega inteligente têm por base sensores de humidade so solo, como 
é o caso do sistema proposto por Goap et al \cite{goap2018an} em que 

\section{Métodos e Metadologias}

O trabalho descrito neste artigo pertende responder a algumas questões que foram 
levantadas após alguma investigação sobre soluções já existentes no que diz respeito 
a sistemas de rega. O sistema proposto permite poupar água? Qual a quantidade de água 
que é possível poupar?  Qual é o custo associado à integração de sensores num sistema 
de rega convencional? Em comparação com um sistema de rega convencional, 
qual a poupança que um sistema de rega inteligente proporciona?

\subsection{O nosso sistema}

O sistema de rega inteligente que propomos faz uso do Arduino MKR 1000 WiFi, 
de um sensor de humidade do solo, modelo 123, de uma breadboard, de vários LEDs 
e uma resistência de 200 ohms. Decidimos usar este modelo do Arduino pela ligação 
à rede por Wi-fi que possui, o que nos permite analisar o sistema sem termos de 
estar no local da instalação, o que podemos fazer através do Arduino Cloud, 
que envia-nos os resultados do sensor que estamos a usar. Caso o utilizador 
queira ver o estado da humidade do solo, pode perceber pelos LEDs que usamos no sistema, 
estes LEDs mostram um feedback visual do estado atual do solo, decidimos representar este 
feedback com 5 LEDs de várias cores que vão desde o verde até ao vermelho, portanto 
se o LED verde estiver acesso quer dizer que o solo está suficientemente humido, se 
o LED laranja estiver acesso quer dizer que solo requer um pouco de água e se o LED
vermelho estiver acesso quer dizer que o solo está seco, e por isso tem que ser regado.

O dispositivo Arduino, como mostra a figura \ref{fig:circuit}, que usamos é o 
modelo MKR 1000 WiFi pois é um modelo com capacidade wifi, o que facilita na 
transmissão dos dados para o utilizador que poderá vê-los no seu smartphone, 
esta interação com o smartphone não foi criada mas é uma possibilidade, bem 
como a utilização de um Raspberry Pi para fazer o tratamento da informação 
recebida pelo Arduino.

O sensor de humidade, ilustrado pela figura \ref{fig:circuit}, é um 
sensor normal para esta função, que tem por base valores entre 0 e 1023, serão usados valores 
incrementais entre os valores mínimo e máximo para fazer uma distinção do grau 
de escassez do solo, a partir destes valores base associamos uma percentagem 
que irá corresponder à humidade do solo, a utilização de percentagem é 
mais user-friendly e consequentemente o sistema dará um feedback mais útil
para o utilizador. Também poderia ser usado um Raspberry Pi para guardar dados do sensor.

\begin{figure}
    \centering
    \includegraphics[scale=0.5]{soil-moisture-circuit-schema.png}
    \caption{Esquema do Circuito}
    \label{fig:circuit}
\end{figure}

\subsubsection{Preço do sistema}

Na Tabela \ref{pricetable} podemos verificar o custo do sistema, 
isto inclui apenas os custos associados aos sensores e equipamento 
estritamente necessário à criação deste sistema, portanto custo associados 
a produtos ou serviços externos não são considerados para esta análise, 
como por exemplo o custo de água gasta, pois estes valores variam ao longo 
do tempo e por região e seria difícil fazer uma estimativa para tal.

\begin{table}[ht]
\centering
\small
\begin{tabular}{|c|c|c|c|c|}
    \hline
    \rowcolor{gray}
    \color{white}Item & \color{white}Preço \\
    \hline
    Arduino MKR 1000 WiFi & 30,70 € \\
    \hline
    Sensor de humidade & 5,40 € \\
    \hline
    Servo motor & 5,10 € \\
    \hline
    LEDs & 13 € (Pack) \\
    \hline
    Resistência & 6 € (Pack de 100) \\
    \hline
    Fios e outros cabos e Breadboard & 7,00 \\
    \hline
    \rowcolor{gray}
    \color{white}Total & \color{white}67,20 € \\
    \hline
\end{tabular}
\vspace{1em}
\caption{Custo do material utilizado}
\label{pricetable}
\end{table}

Como podemos verificar o custo total deste sistema, ou de um sistema de 
rega inteligente similar, não é muito elevado. Apesar de ser difícil estimar 
um valor concreto para o custo de água gasta podemos assumir que seria um custo 
muito mais baixo do que é no momento para um utilizador que não use um sistema de 
rega inteligente, pois este tipo de sistema está desenhado para poupar água. 
Comparando com alguns sistemas de rega inteligente que existem no 
mercado \cite{amazonOrbit} \cite{amazonNetro}, o nosso sistema é mais barato e 
atendendo ao facto de que os preços descritos na tabela \ref{pricetable} incluem packs de 
produtos e a utilização do servo motor neste sistema é para efeitos de simulação 
de uma bomba de água, o preço real do sistema seria ainda mais barato, o preço mais 
baixo poderia ser por volta de 40 €, o que tornaria este sistema atraente para 
potenciais consumidores interessados em sustentabilidade e na redução da sua pegada 
ambiental.

\subsection{Testes ao sistema}

Realizamos vários testes, tanto num ambiente controlado como em ambiente real, desde 
a construção do circuito em simulação até ao momento de entrega final. Começamos 
por fazer testes simulados, para isso usamos a ferramenta Tinkercad \cite{tinkercad} onde 
criamos o circuito e o código associado ao circuito que implementa a lógica 
que o sistema deve seguir, neste ambiente de simulação é possível testar 
o sistema de forma a verificar se o código criado faz realmente aquilo 
que é pretendido.

Num dos testes do primeiro protótipo do sistema, no qual apenas se tinha o microcontrolador
Arduino Uno e o sensor de humidade, foi possível observar que o valor máximo obtido pelo sensor 
era de 1023, quando o recipiente com terra continha demasiada água. Este teste, por ainda 
não se ter implmentado todo o circuito considerou-se ser um teste preliminar.

Posteriormente, realizou-se vários testes no jardim de um dos discentes, no qual os 
resultados obtidos num dos testes, podem ser verificados pelo gráfico da Figura \ref{fig:graphic}. 

Como se pode observar pelo gráfico, os valores iniciais estão comprimidos entre os 280 e os 300, 
havendo uma descida abruta logo após se ter iniciado o teste. Isto deve-se ao facto de se ter começado 
o processo de irrigação. O solo por estar seco começou a absorver água e o sensor perdeu alguma estabilidade.
Após, cerca de 1 minuto, verificou-se uma subida constante no valor da humidade do solo até se atingir 
um valor máximo de 400. Após se atingir este pico, os valores estabilizaram-se, havendo uma pequena variação 
nos valores da humidade. Deixou-se de regar a planta e observou-se que os valores da humidade, começaram 
a diminuir.

\begin{figure}
    \centering
    \includegraphics[scale=0.5]{humidity-test-graph.png}
    \caption{Gráfico dos resultados do teste realizado}
    \label{fig:graphic}
\end{figure}

\section{Conclusão e Trabalho Futuro}

Um dos maiores problemas que vamos enfrentar num futuro próximo é a escassez de água potável,
e sendo a água um bem essencial para a sobrevivência humana, existe uma preocupação 
em criar sistemas que possibilitem uma melhor gestão da água. Por isto mesmo 
é que proposmos este sistema de rega de água inteligente. Apesar de não ser um 
sistema muito robusto e que pela sua natureza funciona melhor em pequena escala, este 
poderia ser uma ponte para a criação de um sistema mais robusto que funciona-se bem 
tanto áreas pequenas como jardins ou estufas, como também em áreas maiores como campos agrículas.

Para trabalho futuro pode ser integrado um Raspberry Pi neste sistema para uma melhor 
gestão de informação. Um dos pontos fracos que o sistema proposto tem é não 
ter em conta a previsão do tempo o que pode tornar este sistema menos eficiente, 
esta previsão de tempo poderia ser integrada no sistema através de um API que 
envia os dados de previsão em tempo real e juntamente com os dados do sensor de 
humidade o sistema tomaria a decisão se regava as plantas ou não.

\bibliographystyle{IEEEtran}
\bibliography{references}

\end{document}
